\documentclass{howto}

\title{Pudding ( a widget system for Soya )}
\release{0.1-0}


\author{Dunk Fordyce, dunk@dunkfordyce.co.uk}

\begin{document}
\maketitle
\ifhtml
\chapter*{Front Matter\label{front}}
\fi

\begin{abstract}
\noindent
This document describes how to use Pudding with Soya. Pudding is a
replacement widget system for Soya's current widget system. 
\end{abstract}

\tableofcontents

\section{What is Pudding?}

Pudding is a widget system primarily for Soya, it could however 
with some tweaking be used for orther libraries such as pyopengl.

\subsection{Why Pudding?}

Pudding was started as a replacement to Soya's current widget module.
The current module, while usefull, is hard to extend.

There are several other opengl UI libraries available but all have
theyre problems or would be complicated to use with soya. 

Pudding has been designed to allow components to be created from a 
core set of base classes. This allows the developer to create any
sort of widget, either, virtually from scratch or from a higher level.

\subsection{Some cake to have and eat}
Here is a minimal example of using pudding for the impatient:

\begin{verbatim}
import soya
import pudding

soya.path.append('data')

soya.init()
pudding.init()

scene = soya.World()

sword_model = soya.Shape.get("sword")
sword = soya.Volume(scene, sword_model)
sword.x = 1
sword.rotate_lateral(90.)

light = soya.Light(scene)
light.set_xyz( .5, 0., 2.)

camera = soya.Camera(scene)
camera.z = 3.

soya.set_root_widget(pudding.core.RootWidget())
soya.root_widget.add_child(camera)

button_bar = pudding.container.HorizontalContainer(soya.root_widget, 
                              left = 10, width= 164, height = 64)
button_bar.set_pos_bottom_right(bottom = 10)
button_bar.anchors = pudding.ANCHOR_BOTTOM

button1 = button_bar.add_child( pudding.control.Button(label = 'Button1'), 
                                pudding.EXPAND_BOTH)
button2 = button_bar.add_child( pudding.control.Button(label = 'Button2'), 
                                pudding.EXPAND_BOTH)

logo = pudding.control.Logo(soya.root_widget, 'mylogo.png')

pudding.idler.Idler(scene).idle()

\end{verbatim}

\section{Software Requirements}

You need to have the following software installed:

\begin{itemize}
  \item  Python 2.3 \url{http://python.org}
  \item  Soya ( and all relevant dependancies ) \url{http://oomadness.tuxfamily.org/en/soya/index.html}
\end{itemize}

Optional software includes:
\begin{itemize}
  \item ElementTree \url{http://effbot.org/zone/element-index.htm}
  \item pycairo \url{http://cairographics.org}
\end{itemize}

\section{Pudding Basics}

This section will introduce the basics of \module{pudding}.

\subsection{Initializing \module{pudding}}

Using pudding is as simple as adding two extra statements to your Soya application.

\begin{verbatim}
import soya
import pudding

soya.init()
pudding.init()
\end{verbatim}

You are now ready to create a \module{pudding} root widget to add components to. 

\subsection{The \class{RootWidget} class}

To use pudding you \emph{must} use \class{pudding.core.RootWidget}.

\begin{verbatim}
# ... initialize soya and pudding

soya.set_root_widget(pudding.core.RootWidget())
\end{verbatim}

To add your camera to the root widget use:

\begin{verbatim}
# ... initialize soya and pudding

scene = soya.World()
camera = soya.Camera(scene)

soya.set_root_widget(pudding.core.RootWidget())

soya.root_widget.add_child(camera)

\end{verbatim}

\subsection{Hello World!}

The infamous hello world script with pudding:

\begin{verbatim}
import soya
import pudding

soya.init()
pudding.init()

scene = soya.World()

camera = soya.Camera(scene)

soya.set_root_widget(pudding.core.RootWidget())
soya.root_widget.add_child(camera)

text = pudding.control.SimpleLabel(soya.root_widget, label = "Hello World!")

pudding.idler.Idler(scene).idle()
\end{verbatim}

\section{Module: \module{pudding} -- Main pudding module}
\declaremodule{standard}{pudding}
\subsection{Functions}
\begin{funcdesc}{init}{style=None}
Intialise \module\{pudding\}. \var\{style\} should be a subclass of 
\class\{pudding.style.Style\}
\end{funcdesc}

\begin{funcdesc}{process_event}{}
This gets the event list from soya and filters it for any events handled 
by widgets. It returns an array with the events that have not been used. 
If you use the \class\{pudding.idler.Idler\} then this function is called in 
\method\{idler.begin\_round\} and the events unprocessed put in 
\var\{idler.events.\}
\end{funcdesc}

\subsection{Classes}
\begin{excdesc}{ConstantError}
Inherits:
\class{PuddingError}
\class{Exception}

Error using a \module\{pudding\} constant

\end{excdesc}

\begin{excdesc}{PuddingError}
Inherits:
\class{Exception}

A \module\{pudding\} exception

\end{excdesc}

\subsection{Constants}
\begin{datadesc}{ALIGN_BOTTOM}
\end{datadesc}
\begin{datadesc}{ALIGN_LEFT}
\end{datadesc}
\begin{datadesc}{ALIGN_RIGHT}
\end{datadesc}
\begin{datadesc}{ALIGN_TOP}
\end{datadesc}
\begin{datadesc}{ANCHOR_ALL}
\end{datadesc}
\begin{datadesc}{ANCHOR_BOTTOM}
\end{datadesc}
\begin{datadesc}{ANCHOR_BOTTOM_LEFT}
\end{datadesc}
\begin{datadesc}{ANCHOR_BOTTOM_RIGHT}
\end{datadesc}
\begin{datadesc}{ANCHOR_LEFT}
\end{datadesc}
\begin{datadesc}{ANCHOR_RIGHT}
\end{datadesc}
\begin{datadesc}{ANCHOR_TOP}
\end{datadesc}
\begin{datadesc}{ANCHOR_TOP_LEFT}
\end{datadesc}
\begin{datadesc}{ANCHOR_TOP_RIGHT}
\end{datadesc}
\begin{datadesc}{BOTTOM_LEFT}
\end{datadesc}
\begin{datadesc}{BOTTOM_RIGHT}
\end{datadesc}
\begin{datadesc}{CENTER_BOTH}
\end{datadesc}
\begin{datadesc}{CENTER_HORIZ}
\end{datadesc}
\begin{datadesc}{CENTER_VERT}
\end{datadesc}
\begin{datadesc}{CLIP_BOTTOM}
\end{datadesc}
\begin{datadesc}{CLIP_LEFT}
\end{datadesc}
\begin{datadesc}{CLIP_NONE}
\end{datadesc}
\begin{datadesc}{CLIP_RIGHT}
\end{datadesc}
\begin{datadesc}{CLIP_TOP}
\end{datadesc}
\begin{datadesc}{CORNERS}
\end{datadesc}
\begin{datadesc}{EXPAND_BOTH}
\end{datadesc}
\begin{datadesc}{EXPAND_HORIZ}
\end{datadesc}
\begin{datadesc}{EXPAND_NONE}
\end{datadesc}
\begin{datadesc}{EXPAND_VERT}
\end{datadesc}
\begin{datadesc}{STYLE}
\end{datadesc}
\begin{datadesc}{TOP_LEFT}
\end{datadesc}
\begin{datadesc}{TOP_RIGHT}
\end{datadesc}

\section{Module: \module{pudding.core} -- Core objects for \module{pudding}}
\declaremodule{standard}{pudding.core}
\subsection{Classes}
\begin{classdesc*}{Base}
Inherits:

The base class for all widgets. Note a Base control doesnt render 
anything to the screen or it does it in a fashion where position and size 
are not relevant. For graphical controls subclass \class\{pudding.Control\}
instead

\begin{memberdesc}{child}
child object
\end{memberdesc}

\begin{memberdesc}{parent}
parent object
\end{memberdesc}

\begin{methoddesc}{advance_time}{self, proportion}
soya advance\_time event
\end{methoddesc}

\begin{methoddesc}{begin_round}{self}
soya begin\_round event
\end{methoddesc}

\begin{methoddesc}{end_round}{self}
soya.end\_round event
\end{methoddesc}

\begin{methoddesc}{on_init}{self}
event occurs at the end of initialisation for user processing
\end{methoddesc}

\begin{methoddesc}{on_set_child}{self, child}
event triggered when the child attribute is set
\end{methoddesc}

\begin{methoddesc}{process_event}{self, event}
process one event. returning False means that the event has not been
handled and should be passed on to other widgets. returning True means
that the event has been handled and the event should no longer be 
propogated
\end{methoddesc}

\end{classdesc*}

\begin{classdesc*}{Control}
Inherits:
\class{Base}

The main graphical base class for all widgets.

\begin{memberdesc}{anchors}

\end{memberdesc}

\begin{memberdesc}{bottom}
distance from the bottom edge of 
the screen to the bottom edge of the control
\end{memberdesc}

\begin{memberdesc}{height}
height of the control
\end{memberdesc}

\begin{memberdesc}{left}
distance from the left edge of 
the screen to the left edge of the control
\end{memberdesc}

\begin{memberdesc}{right}
distance from the right edge of 
the screen to the right edge of the control
\end{memberdesc}

\begin{memberdesc}{screen_bottom}

\end{memberdesc}

\begin{memberdesc}{screen_left}

\end{memberdesc}

\begin{memberdesc}{screen_right}

\end{memberdesc}

\begin{memberdesc}{screen_top}

\end{memberdesc}

\begin{memberdesc}{top}
distrance from the top edge of 
the screen to the top edge of the control
\end{memberdesc}

\begin{memberdesc}{visible}
is the object visible
\end{memberdesc}

\begin{memberdesc}{width}
width of the control
\end{memberdesc}

\begin{methoddesc}{do_anchoring}{self}
move the control based on anchor flags
\end{methoddesc}

\begin{methoddesc}{end_render}{self}
shuts down opengl state
\end{methoddesc}

\begin{methoddesc}{on_hide}{self}
event when the control is made invisible
\end{methoddesc}

\begin{methoddesc}{on_resize}{self}
event when the control is resized
\end{methoddesc}

\begin{methoddesc}{on_show}{self}
event when the control is made visible
\end{methoddesc}

\begin{methoddesc}{process_event}{self, event}
process one event. returning False means that the event has not been
handled and should be passed on to other widgets. returning True means
that the event has been handled and the event should no longer be 
propogated.
\end{methoddesc}

\begin{methoddesc}{render}{self}
render the whole object. setup and take down opengl, render self and 
render all children
\end{methoddesc}

\begin{methoddesc}{render_self}{self}
renders the current object. ie dont render the children, render self
\end{methoddesc}

\begin{methoddesc}{resize}{self, left, top, width, height}
set the position and size of the control
\end{methoddesc}

\begin{methoddesc}{set_pos_bottom_right}{self, right=None, bottom=None}
whereas using .right and .bottom effect the width and height of the 
control this will effect the left and the top
\end{methoddesc}

\begin{methoddesc}{start_render}{self}
sets up opengl state
\end{methoddesc}

\end{classdesc*}

\begin{classdesc*}{InputControl}
This class should be used with multiple inheritance to create
some standard events. call InputControl.process\_event(self,event)
from your widgets process\_event call.

Note the methods on\_mouse*, on\_key\_*, on\_focus and on\_loose\_focus

\begin{memberdesc}{focus}

\end{memberdesc}

\begin{methoddesc}{on_focus}{self}
event triggered when the control gets focus
\end{methoddesc}

\begin{methoddesc}{on_key_down}{self, key, mods}
event triggered when a key is pressed
\end{methoddesc}

\begin{methoddesc}{on_key_up}{self, key, mods}
event triggered when a key is released
\end{methoddesc}

\begin{methoddesc}{on_loose_focus}{self}
event triggered when the control looses focus
\end{methoddesc}

\begin{methoddesc}{on_mouse_down}{self, x, y, button}
event triggered when a mouse button is pressed
\end{methoddesc}

\begin{methoddesc}{on_mouse_over}{self, x, y, buttons}
event triggered when the mouse moves over the control
\end{methoddesc}

\begin{methoddesc}{on_mouse_up}{self, x, y, button}
event triggered when a mouse button is released
\end{methoddesc}

\begin{methoddesc}{process_event}{self, event}
process an individial event and then pass it on the correct event
handler. if that handler returns True the event is assumed to of been 
dealt with
\end{methoddesc}

\begin{methoddesc}{process_mouse_event}{self, event}
process a mouse event. focus is set if the mouse is over the widget.
the event handlers on mouse\_* are called from here
\end{methoddesc}

\end{classdesc*}

\begin{classdesc*}{RootWidget}
Inherits:
\class{Container}
\class{Control}
\class{Base}

The root widget to be used with \module\{pudding\}.

If your display looks incorrect try resizing the window. If that corrects 
the display then you need to call root\_widget.on\_resize() at some 
point before the user gets control.

\begin{methoddesc}{add_child}{self, child}
Add a child to the root widget. \class\{RootWidget\} also accepts cameras
as children altho these are stored in .cameras
\end{methoddesc}

\begin{methoddesc}{on_init}{self}
Declares self.cameras
\end{methoddesc}

\begin{methoddesc}{on_resize}{self}
Resize all cameras and children
\end{methoddesc}

\begin{methoddesc}{start_render}{self}
Load the identity matrix for the root widget
\end{methoddesc}

\begin{methoddesc}{widget_advance_time}{self, proportion}
Called once or more per round
\end{methoddesc}

\begin{methoddesc}{widget_begin_round}{self}
Called at the beginning of every round
\end{methoddesc}

\begin{methoddesc}{widget_end_round}{self}
Called at the end of every round
\end{methoddesc}

\end{classdesc*}


\section{Module: \module{pudding.control} -- most basic widget for pudding}
\declaremodule{standard}{pudding.control}
\subsection{Classes}
\begin{classdesc*}{Button}
Inherits:
\class{Box}
\class{Control}
\class{Base}
\class{InputControl}

A simple button widget. The label is a child SimpleLabel widget.
Note the on\_click method provided

\begin{memberdesc}{label}
label on the button
\end{memberdesc}

\begin{methoddesc}{on_click}{self}
event triggered when the button is "clicked" either by the mouse or the 
keyboard
\end{methoddesc}

\begin{methoddesc}{on_mouse_up}{self, x, y, button}
use the mouse up event handler to implement the on\_click handler
\end{methoddesc}

\begin{methoddesc}{on_resize}{self}
use the resize event to move and resize the buttons child label
\end{methoddesc}

\begin{methoddesc}{render_self}{self}
render the box with current settings
\end{methoddesc}

\end{classdesc*}

\begin{classdesc*}{Console}
Inherits:
\class{VerticalContainer}
\class{Container}
\class{Control}
\class{Base}

A simple console style widget

\begin{methoddesc}{on_focus}{self}
automatically give focus to the input when the console gets focus
\end{methoddesc}

\begin{methoddesc}{on_key_press}{self, key, mods}
allow scrolling thru the buffer
\end{methoddesc}

\begin{methoddesc}{on_loose_focus}{self}
automatically give focus to the input when the console gets focus
\end{methoddesc}

\begin{methoddesc}{on_resize}{self}
update child controls
\end{methoddesc}

\begin{methoddesc}{on_return}{self}
send all input to the output and clear the input ready for more
\end{methoddesc}

\end{classdesc*}

\begin{classdesc*}{Image}
Inherits:
\class{Control}
\class{Base}

A simple image control

\begin{memberdesc}{material}

\end{memberdesc}

\begin{memberdesc}{rotation}

\end{memberdesc}

\begin{memberdesc}{shade}

\end{memberdesc}

\begin{methoddesc}{render_self}{self}
render the image to screen
\end{methoddesc}

\end{classdesc*}

\begin{classdesc*}{Input}
Inherits:
\class{Box}
\class{Control}
\class{Base}
\class{InputControl}

Simple input box using a child SimpleLabel widget. 
Note the on\_value\_changed method

\begin{memberdesc}{cursor}
text used as a cursor. defaults to '_'
\end{memberdesc}

\begin{memberdesc}{prompt}
static text used as a prompt
\end{memberdesc}

\begin{memberdesc}{value}
the text in the input box
\end{memberdesc}

\begin{methoddesc}{clear}{self}
clear the value of the input
\end{methoddesc}

\begin{methoddesc}{on_focus}{self}
append the cursor sybol when focus is gained
\end{methoddesc}

\begin{methoddesc}{on_key_down}{self, key, mods}
process and key strokes and add them to the current value
\end{methoddesc}

\begin{methoddesc}{on_loose_focus}{self}
remove the cursor symbol when the focus is lost
\end{methoddesc}

\begin{methoddesc}{on_resize}{self}
set the position and size of our child label
\end{methoddesc}

\begin{methoddesc}{on_return}{self}
event triggered when the return key is pressed
\end{methoddesc}

\begin{methoddesc}{on_value_changed}{self}
event triggered when the value is changed by the user
\end{methoddesc}

\begin{methoddesc}{set_height_to_font}{self}
set the height of the input control to the height of the font
\end{methoddesc}

\end{classdesc*}

\begin{classdesc*}{Label}
Inherits:
\class{SimpleLabel}
\class{Control}
\class{Base}
\class{InputControl}

Label with events. Created using SimpleLabel and InputControl with 
multiple inheritance

\begin{datadesc}{MAXLEN}
\end{datadesc}
\begin{methoddesc}{process_event}{self, event}
let InputControl class deal with events
\end{methoddesc}

\end{classdesc*}

\begin{classdesc*}{Logo}
Inherits:
\class{Image}
\class{Control}
\class{Base}

Class to display an image in the bottom right corner, usefull for logo's

\end{classdesc*}

\begin{classdesc*}{Panel}
Inherits:
\class{Box}
\class{Control}
\class{Base}
\class{InputControl}

A simple window/panel control with a title. modify the style
class to change the way this is draw

\begin{memberdesc}{label}

\end{memberdesc}

\begin{methoddesc}{process_event}{self, event}
default event handling
\end{methoddesc}

\begin{methoddesc}{render_self}{self}
render the box with current settings
\end{methoddesc}

\end{classdesc*}

\begin{classdesc*}{PrePostLabel}
Inherits:
\class{SimpleLabel}
\class{Control}
\class{Base}

A label with static pre/post -fix

\begin{datadesc}{MAXLEN}
\end{datadesc}
\begin{memberdesc}{post}

\end{memberdesc}

\begin{memberdesc}{pre}

\end{memberdesc}

\begin{methoddesc}{set_display_text}{self, text}
set the display text with the pre and post strings
\end{methoddesc}

\end{classdesc*}

\begin{classdesc*}{SimpleLabel}
Inherits:
\class{Control}
\class{Base}

A simple, unresponsive label widget

\begin{datadesc}{MAXLEN}
\end{datadesc}
\begin{memberdesc}{autosize}
should the label automatically adjust its size 
to accomodate all the text
\end{memberdesc}

\begin{memberdesc}{clip}
if there is too much text do we clip the left 
or right. must be pudding.core.[pudding.CLIP_LEFT | 
pudding.CLIP_RIGHT]
\end{memberdesc}

\begin{memberdesc}{color}
color of the text
\end{memberdesc}

\begin{memberdesc}{font}
font used for rendering
\end{memberdesc}

\begin{memberdesc}{label}
text displayed
\end{memberdesc}

\begin{memberdesc}{wrap}

\end{memberdesc}

\begin{methoddesc}{on_resize}{self}
update position with anchoring and apply wrapping/clipping
\end{methoddesc}

\begin{methoddesc}{on_set_label}{self}
event triggered when the label is changed
\end{methoddesc}

\begin{methoddesc}{render_self}{self}
draw the text with the current settings
\end{methoddesc}

\begin{methoddesc}{set_display_text}{self, text}
get the text we should actually display. usefull if you want to add a
constant string or perform some processing before the string gets 
wrapped or clipped or whatever
\end{methoddesc}

\begin{methoddesc}{update}{self}
refresh settings based on clip and autoresize etc
\end{methoddesc}

\end{classdesc*}


\section{Module: \module{pudding.container} -- containers for pudding}
\declaremodule{standard}{pudding.container}
\subsection{Classes}
\begin{classdesc*}{HorizontalContainer}
Inherits:
\class{Container}
\class{Control}
\class{Base}

class to resize all children in a row

\begin{methoddesc}{on_resize}{self}
resize all children into a row
\end{methoddesc}

\end{classdesc*}

\begin{classdesc*}{VerticalContainer}
Inherits:
\class{Container}
\class{Control}
\class{Base}

class to resize all children in a column

\begin{methoddesc}{on_resize}{self}
resize all children into a column
\end{methoddesc}

\end{classdesc*}


\section{Module: \module{pudding.idler} -- a simple replacement idler for soya}
\declaremodule{standard}{pudding.idler}
\subsection{Classes}
\begin{classdesc*}{Idler}
Inherits:
\class{Idler}

Simple replacement for the soya.Idler that calls pudding.process\_event
in begin\_round and places all unhandled events into idler.events.

\begin{methoddesc}{begin_round}{self}
call pudding.process event and put all events in self.events so the 
"game" can handle other events
\end{methoddesc}

\begin{methoddesc}{idle}{self}
resize all widgets and start the idler
\end{methoddesc}

\end{classdesc*}


\section{\module{pudding.sysfont} -- sysfont, used in the font module to find system fonts}




%\section{This is an Appendix}

%To create an appendix in a Python HOWTO document, use markup like
%this:

%\begin{verbatim}
%\appendix

%\section{This is an Appendix}

%To create an appendix in a Python HOWTO document, ....


%\section{This is another}

%Just add another \section{}, but don't say \appendix again.
%\end{verbatim}


\end{document}
